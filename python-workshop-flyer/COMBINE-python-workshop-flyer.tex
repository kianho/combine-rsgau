\documentclass[12pt,a4paper]{article}
\usepackage[noheadfoot,nomarginpar,top=10mm,bottom=10mm,left=10mm,right=10mm]{geometry}
\usepackage[svgnames,x11names,dvipsnames]{xcolor}
\usepackage{tikz}
\usepackage{lipsum}
\usepackage{varwidth}
\usepackage{makecell}
\usepackage{graphicx}
\usepackage{setspace}
\usepackage{verbatim}
\usepackage{tabu}
\usepackage[noframe]{showframe}

\usepackage{lmodern}
\renewcommand*\familydefault{\sfdefault}
\usepackage[T1]{fontenc}

\begin{document}
    %   \begin{tikzpicture}[remember picture,overlay]
    %     \shade[top color=LightSteelBlue, bottom color=white]
    %     (current page.north west) rectangle
    %     (current page.south east)  ;
    %   \end{tikzpicture}
    \thispagestyle{empty}
    \noindent\\
    \vfill
    \begin{minipage}[c]{0.9\textwidth}
        \centering
        \extrarowsep=1ex
        \begin{tabu} to 0.9\linewidth {X[1]X[3]}
            \includegraphics[height=40mm]{./images/python-logo.png} &  \\
            & COMBINE proudly presents\\[0.5ex]
            &\raggedright {\LARGE\scshape\bfseries Python for the Life
        Sciences}\\[3ex]
            \raggedleft\textbf{Date} & \_\_\_ September, 2013\\
            \raggedleft\textbf{Time} & TBD\\
            \raggedleft\textbf{Venue} & TBD\\
            \raggedleft\textbf{RSVP} & TBD (Max.~of 20 places)\\
            \raggedleft\textbf{Instructor} & Dr.~Bernie Pope, VLSCI\\
            \raggedleft\textbf{Description} & 
        Python is a general-purpose programming language suitable for
        programmers of all abilities due to its distinctly clear and readable
        syntax. It is used in a wide variety disciplines and is supported by an
        extensive range of open-source libraries. One of its many strengths is
        its ability to perform ad-hoc analysis and visualisation of large
        datasets common to the computational life sciences; whether it be
        sequence, network, genomic, structural, or molecular data.\\[1ex]
        \raggedleft\textbf{About us}&\raggedright 
            COMBINE is a student-run organisation for researchers in
            computational biology, bioinformatics, and related fields.\linebreak
            
            We aim to bring together students and early-career researchers from
            the computational and life sciences for networking, collaboration,
            and professional development.\linebreak

            COMBINE is sponsored by the ICT for the Life Sciences Forum.
        \end{tabu}
    \end{minipage}\\
    \vspace{40mm}
    \vfill
\end{document}
