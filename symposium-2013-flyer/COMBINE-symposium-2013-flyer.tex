\documentclass[12pt,]{article}
\usepackage[T1]{fontenc}
\usepackage{lmodern}
\usepackage{amssymb,amsmath}
\usepackage{ifxetex,ifluatex}
\usepackage{fixltx2e} % provides \textsubscript
% use microtype if available
\IfFileExists{microtype.sty}{\usepackage{microtype}}{}
% use upquote if available, for straight quotes in verbatim environments
\IfFileExists{upquote.sty}{\usepackage{upquote}}{}
\ifnum 0\ifxetex 1\fi\ifluatex 1\fi=0 % if pdftex
  \usepackage[utf8]{inputenc}
\else % if luatex or xelatex
  \usepackage{fontspec}
  \ifxetex
    \usepackage{xltxtra,xunicode}
  \fi
  \defaultfontfeatures{Mapping=tex-text,Scale=MatchLowercase}
  \newcommand{\euro}{€}
\fi
\usepackage[a3paper,margin=35mm]{geometry}
\usepackage{longtable}
\ifxetex
  \usepackage[setpagesize=false, % page size defined by xetex
              unicode=false, % unicode breaks when used with xetex
              xetex]{hyperref}
\else
  \usepackage[unicode=true]{hyperref}
\fi
\hypersetup{breaklinks=true,
            bookmarks=true,
            pdfauthor={},
            pdftitle={},
            colorlinks=true,
            urlcolor=blue,
            linkcolor=magenta,
            pdfborder={0 0 0}}
\urlstyle{same}  % don't use monospace font for urls
\setlength{\parindent}{0pt}
\setlength{\parskip}{6pt plus 2pt minus 1pt}
\setlength{\emergencystretch}{3em}  % prevent overfull lines
\setcounter{secnumdepth}{0}
% vim:ft=tex

\usepackage{lmodern}
\renewcommand*\familydefault{\sfdefault}
\usepackage[T1]{fontenc}
\usepackage{calc}
\usepackage{graphicx}
\usepackage[dvipsnames,usenames]{color}
\usepackage{tikz}
\usepackage[all,top]{background}
    \SetBgContents{
       \begin{tikzpicture}[remember picture,overlay]
          \shade[top color=blue!25!white,middle color=white, bottom color=white]
          (-0.5\paperwidth,1ex) rectangle (\paperwidth,-0.05\paperheight) ;
          \shade[bottom color=black!25!white,middle color=white,top color=white]
          (-0.5\paperwidth,-0.95\paperheight) rectangle
          (\paperwidth,-\paperheight) ;
        \end{tikzpicture}
    }
    \SetBgOpacity{1}
    \SetBgScale{1}
    \SetBgAngle{0}
    \pagestyle{empty}
\usepackage{titlesec}
    \titleformat*{\section}{\Huge\bfseries}
\usepackage{wrapfig}

\author{}
\date{}

\begin{document}

\newpage
\null
\vfill

\section{COMBINE WORKSHOP SERIES}

\vspace{-2ex}

\subsection{\Large \emph{Python for the Life Sciences}}

\vspace{-2ex}

\begin{longtable}[c]{@{}ll@{}}
\hline\noalign{\medskip}
\begin{minipage}[t]{0.21\columnwidth}\raggedright
\emph{Presenter}
\end{minipage} & \begin{minipage}[t]{0.79\columnwidth}\raggedright
Dr.~Bernie Pope, \emph{Specialist Programmer}\\
\end{minipage}
\\\noalign{\medskip}
\begin{minipage}[t]{0.21\columnwidth}\raggedright
\emph{Date \& Time}
\end{minipage} & \begin{minipage}[t]{0.79\columnwidth}\raggedright
Thursday, 26 September, 2013\\2:00pm to 5:00pm
\end{minipage}
\\\noalign{\medskip}
\begin{minipage}[t]{0.21\columnwidth}\raggedright
\emph{Venue}
\end{minipage} & \begin{minipage}[t]{0.79\columnwidth}\raggedright
Room 546, Building 263\\Melbourne Graduate School of Education\\234
Queensberry Street\\The University of Melbourne\\
\end{minipage}
\\\noalign{\medskip}
\begin{minipage}[t]{0.21\columnwidth}\raggedright
\end{minipage} & \begin{minipage}[t]{0.79\columnwidth}\raggedright
Campus Maps Ref:
\href{http://maps.unimelb.edu.au/parkville/building/263}{U17}\\Google
Maps: \href{http://goo.gl/maps/sZ3zm}{goo.gl/maps/sZ3zm}
\end{minipage}
\\\noalign{\medskip}
\begin{minipage}[t]{0.21\columnwidth}\raggedright
\emph{Registration}
\end{minipage} & \begin{minipage}[t]{0.79\columnwidth}\raggedright
Sign up at \href{http://goo.gl/lht4mQ}{goo.gl/lht4mQ}, registration is
\emph{FREE} but limited to 20 places.
\end{minipage}
\\\noalign{\medskip}
\begin{minipage}[t]{0.21\columnwidth}\raggedright
\emph{What to Bring}
\end{minipage} & \begin{minipage}[t]{0.79\columnwidth}\raggedright
Your laptop.
\end{minipage}
\\\noalign{\medskip}
\begin{minipage}[t]{0.21\columnwidth}\raggedright
\emph{Prerequisites}
\end{minipage} & \begin{minipage}[t]{0.79\columnwidth}\raggedright
A basic proficiency in any programming language is preferred.
\end{minipage}
\\\noalign{\medskip}
\begin{minipage}[t]{0.21\columnwidth}\raggedright
\emph{Enquiries}
\end{minipage} & \begin{minipage}[t]{0.79\columnwidth}\raggedright
\href{mailto:workshops@combine.org.au}{workshops@combine.org.au}
\end{minipage}
\\\noalign{\medskip}
\hline
\end{longtable}

\subsubsection{Workshop Description}

In this workshop, we provide a hands-on introduction to the fundamentals
of the Python programming language
(\href{http://www.python.org}{python.org}) and its data processing
features through worked-examples involving bioinformatics datasets.

Upon completion, you will have gained a foundational proficiency in
Python, enabling you to use some of its powerful (and
programmer-friendly) features in research-related tasks.

We are aiming to cover the following topics:

A review of programming fundamentals (e.g.~variables, loops, conditions,
procedures, and data types), principles of good programming style,
introduction to bioinformatics libraries, and manipulating
bioinformatics datasets.

If you have a specific topic or problem in your research programming
that you would like covered, please email us at
\href{mailto:workshops@combine.org.au}{workshops@combine.org.au}.

\subsubsection{About the Presenter}

Dr.~Bernie Pope is a specialist programmer at the VLSCI, a
high-performance computing (HPC) facility for life sciences research. He
is also an Honorary Fellow and Lecturer for
\emph{\href{https://handbook.unimelb.edu.au/view/2013/COMP10001}{COMP10001
Foundations in Computing}} in the Department of Computing and
Information Systems at the University of Melbourne.

His research interests include programming language implementation and
theory, especially functional programming languages such as Haskell.
Recently focussing on bioinformatics and computational biology, Dr Pope
teaches several of VLSCI's programming skills courses, helping
scientists unlock the potential of HPC facilities for their everyday
research.

For more details, please visit his website at
\href{http://www.berniepope.id.au}{berniepope.id.au}.

\subsubsection{About Us}

\href{http://www.combine.org.au}{COMBINE} is the official
\href{http://www.iscbsc.org/content/regional-student-groups}{ISCB
Regional Student Group} for Australia, a student-run organisation for
researchers in computational biology, bioinformatics, and related
fields.

For more information about COMBINE, upcoming events, or to become a
volunteer, please visit our website at
\href{http://www.combine.org.au}{combine.org.au}.

We are proudly sponsored by:

 \vspace{2ex}
\includegraphics[height=20mm]{./images/ICT-for-Life-Sciences-Forum-logo.png}\quad
\includegraphics[height=20mm]{./images/ISCBSC-logo.png}

\begin{itemize}
\itemsep1pt\parskip0pt\parsep0pt
\item
  \href{http://www.ict4lifesciences.org.au}{ICT for Life Sciences Forum}
\item
  \href{http://www.iscb.org}{International Society for Computational
  Biology (ISCB)}
\end{itemize}

\vfill

\end{document}
