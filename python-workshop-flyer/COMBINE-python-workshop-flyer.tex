\documentclass[10pt]{article}
\usepackage[noheadfoot,nomarginpar,vmargin={10mm,10mm},hmargin={15mm,15mm},a4paper]{geometry}
\usepackage{booktabs}
\usepackage{varwidth}
\usepackage{lipsum}
\usepackage{graphicx}
\usepackage[hidelinks]{hyperref}
\usepackage{tabularx}
\usepackage{tabu}
\usepackage[noframe]{showframe}
\usepackage{pbox}
\usepackage{lmodern}
\renewcommand*\familydefault{\sfdefault}
\usepackage[T1]{fontenc}


\begin{document}
\newpage
\pagestyle{empty}
\null
\noindent
\begin{minipage}[t]{\linewidth}
\tabulinesep=1.7ex
\begin{tabu}{rX}
    \toprule
%            & \noindent \raggedleft\includegraphics[height=25mm]{./python-logo-master-v3-TM-flattened.png}\\
    & \begin{minipage}[t]{\linewidth}
    \large{\textbf{COMBINE}} Workshop Series\\
    \Huge{\bfseries Python for the Life Sciences}
\end{minipage} \\
    \midrule
    \textbf{Date} & Thursday, 26 September, 2013 \par
                        2:00pm to 5:00pm \\
    \textbf{Venue} & \begin{minipage}[t]{0.6\linewidth}
                    Room 546, Building 263 \\
                    Melbourne Graduate School of Education \\
                    234 Queensberry Street \\
                    The University of Melbourne
                \end{minipage}\quad
                \begin{varwidth}[t]{0.3\linewidth}
                    Campus Map Ref: \\
                    Google Maps:
                \end{varwidth}\\
                \textbf{Registration} & \begin{varwidth}[t]{\linewidth}
                Sign up at \href{TBD}{\bfseries TBD}, registration is \emph{FREE} but
                limited to TBD places.
            \end{varwidth}\\
    \textbf{What to Bring} & Your laptop.\\
    \textbf{Prerequisites} & A basic proficiency in any programming language is
            preferred.\\
    \textbf{Description} & \begin{varwidth}[t]{\linewidth}
                \raggedright
                In this workshop, we provide a hands-on introduction to the
                fundamentals of Python and its data processing features through
                worked-examples involving bioinformatics datasets.\par
                \medskip
                Upon completion of the workshop, you will have gained a
                foundational proficiency in Python, enabling you to use
                some of its powerful (and programmer-friendly) features
                in research-related tasks.\par
                \medskip
                The following topics will be covered:
                \begin{itemize}
                    \item[] A revision of programming fundamentals (e.g.
                        variables, loops, conditions, procedures, and data
                        types).
                    \item[]
                        Principles of good programming style.
                    \item[]
                        Introduction to bioinformatics libraries.
                    \item[]
                        Manipulating bioinformatics datasets.
                \end{itemize}
                If you have a specific topic or problem in your research
                programming that you would like covered, please email us at
                \href{TBD}{\bfseries TBD}.
                \end{varwidth} \\
       \begin{varwidth}[t]{\linewidth}
           \textbf{What is\\ Python?}
       \end{varwidth} &
                \begin{varwidth}[t]{\linewidth}
                    \raggedright
                Python is a general-purpose programming language suitable for
                programmers of all abilities due to its distinctly clear and
                readable syntax. It is used in a wide variety disciplines and is
                supported by an extensive range of open-source libraries.\par
                \medskip
                One of its many strengths is its ability to perform ad-hoc
                analysis and visualisation of large datasets common to the
                computational life sciences; whether it be sequence, network,
                genomic, or structural data.
                               \end{varwidth} \\
        \textbf{Instructor} & \begin{varwidth}[t]{\linewidth}
                               Dr.~Bernie Pope, \emph{Specialist Programmer}\\
                        Victorian Life Sciences Computation Initiative (VLSCI) \par
                    \bigskip
                    \raggedright
                    Bernie is is a specialist programmer at the VLSCI, a
                    high-performance computing facility for life sciences
                    and an honorary fellow of Computing and Information
                    Systems at the University of Melbourne,
                    Australia.\par
                    \medskip
                    His interests in computing include programming language
                    implementation and theory, especially functional
                    programming languages such as Haskell; and more
                    recently, bioinformatics and computational biology.\par
                    \medskip
                    For more details, visit his website at
                    \href{http://berniepope.id.au/}{http://berniepope.id.au/}.
                \end{varwidth}\\
                \midrule
       \begin{varwidth}[t]{\linewidth}
           \textbf{About Us}
       \end{varwidth} &
    \noindent\begin{minipage}[t]{\linewidth}
        \emph{COMBINE} is a student-run organisation for researchers in computational
        biology, bioinformatics, and related fields. We are also the official ISCB Regional Student
        Group for Australia.\par
        \medskip
        We are proudly sponsored by:\par
        \begin{minipage}[t]{\linewidth}
            \centering
        \end{minipage}
    \end{minipage}\\
    \bottomrule
    \end{tabu}
\end{minipage}
\end{document}
